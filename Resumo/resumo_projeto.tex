\documentclass[12pt, 
%openright,	
oneside,		  
a4paper,			
english,			
brazil				 
]{article}
%\usepackage{abnt2cite}
\usepackage[brazil]{babel}
\usepackage{comment}
%\usepackage[latin1]{inputenc} %  os acentos s\~a\^a£o digitados diretamente pelo teclado
\usepackage{graphicx} % comando para importar figura de outro programa
\usepackage{psfrag}
\usepackage{tabularx}
\usepackage{amssymb}  %comando para usar simbolos, como /therefore  e /mathbb{R
\usepackage[utf8]{inputenc}
\usepackage{amsthm,amsfonts,latexsym}
\usepackage{amsmath,amssymb}
\usepackage[usenames]{color}    %comando para cores
\setlength{\topmargin}{-1.0cm}  %-1.0cm
\setlength{\oddsidemargin}{0.0cm} %0.0cm
\setlength{\evensidemargin}{-2.0cm} %-2.0cm
\setlength{\textheight}{24.cm}
\setlength{\textwidth}{17.cm} %17cm
%\pagestyle{empty} \setlength{\textwidth}{16cm}
\usepackage{indentfirst}
\usepackage{multirow}
\usepackage[normalem]{ulem}
\usepackage[table,xcdraw]{xcolor}
\usepackage{graphicx}
\usepackage{colortbl}
\usepackage{hyperref}
\usepackage{hhline}
\usepackage{indentfirst}
\usepackage{enumerate}
\usepackage[alf]{abntex2cite}
\renewcommand{\baselinestretch}{1.5}
\setlength{\textwidth}{16cm}
\usepackage{fancyhdr}

\begin{document}
\thispagestyle{empty}

\centerline{\Large \bf Análise Exploratória dos Impactos da COVID-19}

\centerline{\Large \bf
no estado de São Paulo}
\smallskip


\begin{center}
\bigskip\bigskip\bigskip\bigskip\bigskip\bigskip
{\large{\bf Instituição}: IBB/Unesp - Botucatu}\\

{\large Departamento de Bioestatística, Biologia Vegetal, Parasitologia e Zoologia}  \\[.1cm]
{\large Mestrado em Biometria}  \\[.1cm]
\end{center}
%\medskip
\bigskip\bigskip\bigskip\bigskip\bigskip\bigskip
%\centerline{\large \bf N\'ivel: Iniciaç\~ao Cient\'ifica}

\bigskip

\begin{center}
{{\bf Discentes}: Maicon Centner Germano, Fernando Andrade, Douglas de Aquino Carrega}\\[.1cm]
Programa de Pós-graduação em Biometria, Unesp - Botucatu.\\[.1cm]

{{\bf Professor}: Thomas N. Vilches}\\[.1cm]

\end{center}

\newpage

\bigskip\bigskip\bigskip

\section{Data Set}
O dataset "dados-covid-sp" é uma fonte de dados abertos mantida pela SEADE-R (Fundação Sistema Estadual de Análise de Dados do Estado de São Paulo) e disponibilizada no GitHub. Ele contém informações detalhadas sobre a evolução da COVID-19 no estado de São Paulo.

Este conjunto de dados é atualizado regularmente e abrange um período específico, fornecendo uma visão abrangente dos impactos da pandemia na região. Os dados são fornecidos em formato tabular, com cada linha representando um registro individual.

As colunas do dataset incluem informações relevantes, como data de notificação, município, faixa etária, sexo, condição (caso confirmado, óbito, recuperado), além de outras variáveis relacionadas à pandemia. Essas informações permitem uma análise aprofundada dos aspectos demográficos, geográficos e temporais da propagação da doença em São Paulo.

O dataset é uma valiosa fonte de informação para pesquisadores, cientistas de dados e profissionais da saúde interessados em entender melhor a disseminação da COVID-19, identificar padrões e tendências, além de contribuir para a tomada de decisões estratégicas relacionadas ao controle e prevenção da doença no estado de São Paulo.

Ao explorar esse dataset, é possível realizar uma variedade de análises, incluindo análise exploratória de dados, visualizações gráficas, análises temporais e espaciais, além de investigar possíveis correlações com outras variáveis relevantes, como medidas de controle da pandemia e taxas de vacinação.

No geral, o dataset "dados-covid-sp" é uma fonte confiável e abrangente que oferece insights valiosos para compreender o impacto da COVID-19 no estado de São Paulo e auxiliar no desenvolvimento de estratégias eficazes para combater a pandemia.

\section{Objetivo}

%insira aqui o conteúdo do objetivo


\section{Metodologia}

\begin{itemize}
    \item Coleta e Preparação dos Dados: Baixamos o dataset "dados-covid-sp" e realizamos a limpeza dos dados.

    \item Análise Exploratória dos Dados: Exploramos a estrutura dos dados e identificamos variáveis demográficas e socioeconômicas relevantes.

    \item Visualização Geográfica: Utilizamos a biblioteca ggplot2 para criar mapas interativos que mostram a distribuição geográfica dos casos de COVID-19 em São Paulo.

    \item Análise de Correlação: Calculamos correlações entre os casos de COVID-19 e as variáveis demográficas ou socioeconômicas.

    \item Análise Espacial: Aplicamos técnicas de análise espacial para identificar padrões geográficos nos casos de COVID-19.

    \item Interpretação dos Resultados: Analisamos os resultados obtidos e interpretamos as visualizações e correlações encontradas.
\end{itemize}

%aqui vai nossa metodologia

\end{document}